%
%		AFIT THESIS MACRO PACKAGE DOCUMENTATION
%                      for version 2.7 of afthesis.cls
%
%
% This file shows the directions of preparing your thesis using the 
% `afthesis' LaTeX document class. This class is an extremely modifed
% `report' document class with new commands added and some old
% commands modified to produce the proper format for the Air Force 
% Institute of Technology thesis or dissertation.
%
% To keep everything simple, this file is designed so that you can use a
% a copy of this file as your LaTeX input file after replacing the
% necessary data by your own data, and inserting your text in the proper
% positions. Inserting text can be done by:
%	-- actually typing the text, or
%	-- using LaTeX \input or \include command
% in the designated position.
% Note that your LaTeX input file name should have the .tex extension, 
% as are the files to be \input'd or \include'd.
%
% The commands \input{foo} inside mythesis.tex will have the effect as 
% if the contents of foo.tex is inserted in the position where the 
% \input command is encountered. To run LaTeX, use the 
% command
%
%	latex mythesis 
%
% To be able to write the inserted text correctly, you are supposed to 
% know basic LaTeX.  All you need to know about LaTeX is written in 
% Leslie Lamport's
% `LaTeX: A Document Preparation System' (Addison-Wesley 1986), which is
% available in local bookstores. 

\documentclass[11pt]{afthesis}

%\documentclass[10pt]{afthesis}  %if you want 10pt instead of 11
%\documentclass[12pt]{afthesis}  %if you want 12pt instead of 11


%\dissertation   % print DISSERTATION instead of THESIS on the title

% Number by chapter ?
%
% You may specify the numbering in your thesis/dissertation to be 
% chapter numbering instead of the default of sequential numbering.  
% If you select this option you will get pages, figures, tables, and 
% equations numbered by chapter (e.g., Table 2.3, Figure 3.4, 
% page numbers 2-40, A-1)
% To not get chapter numbering add a `%' character to the beginning of 
% the next line 

\numberbychapter

% Print section numbers ?
%
% You may select not to have section (and subsection, etc.) numbers 
% printed in the text and in the table of contents. In fact this is 
% the way the AFIT thesis guide shows it, but I like section numbers 
% so I made having section numbers the default.
% To get no section numbers remove the `%' character in the beginning 
% of the next line

%\nosectionnumbers

% Type of empasis ?
%
% You may select to have your emphasized text (like chapter and section
% headings, book titles, foreign phrases, etc.) underlined instead of
% set in an italic font.  By selecting this option, appropriate titles
% are automatically changed, plus anytime you use the command {\em ...}
% you will get underlined text, instead of italic text.  NOTE: this 
% option is not recommended for typeset quality documents. It is here 
% only for those who are old fashioned, type-writer personalities.
% To get underlining instead of italics remove the `%' character in the
% beginning of the next line

%\underlineoption

% Flyleaf frame ?
% 
% You can select to have a 4in by 2in frame put around your flyleaf 
% material.  This makes it look a little nicer if you don't have the 
% cover with the hole in it.
% To get a flyleaf frame remove the `%' character in the beginning of
% the next line.

%\flyleafframe

% Line spacing
%
% The default line spacing is to doublespace except in quotations, 
% quotes, and the bibliography.  This approximates the spacing you 
% get if you "doublespace" on a typewriter.  If you want to change 
% the line spacing use the command \spacing{n} where n is a real 
% number at the start of the document and \endspacing at the end 
% of the document. Use 1 for n to get singlespacing, 1.5 for space 
% and a half, etc.  If you want to change the linespacing to 
% singlespace for a particular section of text, you can use the 
% singlespace environment bracketting your text with 
% \begin{singlespace} and  \end{singlespace} \spacing{2} is the 
% default line spacing for the thesis in 10pt \spacing{1.5} is the 
% default line spacing for the thesis in 11pt and 12pt
%
%  THE ABOVE LINE SPACING INFORMATION IS NOT QUITE ACCURATE
%
% Data of author and thesis: The following data will be used throughout
% your thesis when they are needed. Please replace the dots in the
% commands by your own data. For some commands, the specified default
% value will be assumed when the command is omitted.  For a two author
% thesis, specify the command \twoauthor and then enter the appropriate
% additional fields.  Remember you will need two vitas specified in
% author order.  Authors should be specfied in alphabetical order.

\author{Justin Fletcher}


\rank{First Lieutenant, USAF}

\title{SIMULATED QUANTUM ANISOTROPIC ANNEALING APPLIED TO ARTIFICIAL NEURAL NETWORK WEIGHT SELECTION}

%\flytitle{...}
%
% Remove the % and replace the dots in the above command with your
% thesis title as it should be on the flyleaf.  This is only needed
% if your flyleaf title has different line breaks 
% (because it must fit in 4 inches)
% than the way it appears on the title page and on the first page.
        
\designator{AFIT/GE/ENG/16-..}
	%
	% Replace the dots in the above command with the thesis or dissertation
	% designator. For example, `AFIT/GCS/ENG/87-5'.

%\distribution{...}
	%
	% Replace the dots in the above command with the distribution
	% statement for your thesis.  The default if commented out is 
	% `Approved for public release; distribution unlimited'.

\previousdegrees{B.S. CEC}
%\previousdegreestwo{...} %uncomment for twoauthor option
	%
	% Replace the dots in the above command with the 
	% abbreviated form of your previous degree(s), e.g., B.S. or B.A., M.S.
        % Leave this command out if you have no previous degrees.

\degree{Master of Science in Computer Science}
	%
	% The degree sought as determined by your program. 
	% For example, `\degree{Master of Science}', or
	% `\degree{Master of Science in Electrical Engineering}'.
	% The default value is `Doctor of Philosophy' for dissertation.

\graduationdate{June, 2016}
	%
	% Replace the dots in the above command with the 
	% graduation date, in the form as `\graduationdate{May, 1986}'.
	% The default value is guessed according to the time of running LaTeX.

\address{452 Orchard Drive\\Oakwood, Ohio 45419}
%\addresstwo{...\\...} %uncomment for twoauthor option
	%
	% Replace the dots in the above command with your permanent address.
	% Use \\ to separate address lines.  This is used in the Vita.
	% e.g., `\address{4533 Avenue A\\ Austin, Texas 78751}'.

\school{Electrical and Computer Engineering}
	%
	% Replace the dots in the above command with the name of your school.  
	% For example, `\school{School of Engineering}'

%**********for dissertations only, remove the % signs and add the data
%\dean{...}
	% Needed for disserations only.
	% The name of your dean, e.g., `\dean{Robert A. Calico, Jr}

\committee{Dr. Michael J. Mendenhall\\Thesis Advisor,
	   Dr. Gilbert L. Peterson\\Committee Member,
	   Capt. Charlton D. Lewis\\Committee Member}
	% 
	% The default value is 5 for dissertation. 

\begin{document}
	
% The following commands will automatically generate headings, adjust
% vertical spacings, break pages, etc.
% You should probably leave all of these prefatory pages commented out
% or in a \include file until your thesis is ready for final draft

\flyleaf      			% Generates the flyleaf.

\disclaimerpage                 % Produces the disclaimer page

\titlepage			% Produces the title page.

\approvalpage                   % Produces the approvalpage

\begin{preface}
	%
	%Insert the text of your preface here. Your name will appear
	%automatically. If this is an acknowledgments section instead of 
        %preface, use \begin{acknowledgments} and \end{acknowledgments}
	% instead.
	%
\end{preface}

\tableofcontents	% Table of Contents will be automatically
					% generated and placed here.

\listoffigures  	% List of Figures, List of Tables, and List of
\listoftables		% Symbols will be placed here, if applicable.
\listofsymbols      % Do not use these if you have no such lists.
% To put symbols in the list use command \symbol[#1]{#2}
% where #2 is the symbol and #1 is the definition to be put in the
% list of symbols. The symbol is also automatically put in
% your text.  Leave out [#1] if you don't want a definition.

\listofabbreviations

\abbreviation[Artificial Neural Network]{ANN}
% similar to the list of symbols.  Use command \abbreviation[#1]{#2}
% where #2 is the abbreviation and #1 is the definition to be put in the
% list of abbreviations. The abbreviation is also automatically put in
% your text.  Leave out [#1] if you don't want a definition.

\begin{abstract}

% DO: Write abstract.	

\end{abstract}


\chapter{Introduction}

% DO: Write introduction.	
% The first chapter. \chapter command is of the form \chapter[..]{..} or \chapter{..} where {chapter heading} and [entry in table of contents].

% Do this after the thesis is written. I created an algorithm which simulates quantum tunneling through an error manifold. I applied this algorithm to the problem of selecting weight values for a neural network. It works well for small data sets, but is slower when dimensionality of the data set on which the network is trained is very large. 

In chapter three the traversal a error manifold in the problem configuration space is discussed at length. Several traversal methodologies are proposed and evaluated. 


% Important: If your chapter heading consists of more than one lines, it will be automatically broken into separate lines. However, if you don't like the way LaTeX breaks the chapter heading into lines, use `\newheadline' command to break lines. NEVER USE \\ IN SECTIONAL (E.G., CHAPTER, SECTION, SUBSECTION) HEADINGS!!!!!!!!

\chapter{Background} % This is Chapter 2.



This chapter serves as an comprehensive review of the physical and computational concept material to the topic of this thesis. Relevant work in simulated annealing will be summarized. Next, artificial neural networks and their application will be discussed. The chapter concludes with a very brief overview of the quantum mechanical concepts employed throughout the document. Finally, the notation and terminology conventions to be used throughout the rest of the document are presented.

\section{Artificial Neural Networks}

Artificial Neural Networks (ANN) 

\subsection{Back Propagation Training}


\section{Related Works in Simulated Annealing}

Simulated annealing (SA) is a stochastic optimization algorithm which can be used to find the global minimum of a cost function mapped from the configurations of a combinatorial optimization problem. The concept of simulated annealing was introduced in (Kirkpatrick) as an application of the methods of statistical mechanics to the problem of discrete combinatorial optimization. Specifically, simulated annealing is an extension of the Metropolis-Hastings (Metropolis) algorithm which is used to estimate the ground energy state of a many-body systems at thermal equilibrium. Kirkpatrick et al. applied the Metropolis-Hastings algorithm sequentially, with decreasing temperature values in order to approximate a solid slowly cooling to low temperatures. Later work by Geoff (Geoff) and Cortana et al. (Cortana) extended simulated annealing to the continuous domain. The basic simulated annealing algorithm, as it will be used in this paper, is presented in [algo1]. 

[algo1]

The physical inspiration for simulated annealing. 

Simulated annealing has been successfully applied to numerous domains.

In the parlance of simulated annealing (Kirkpatrick) a system at its maximum temperature is said to be \textit{melted}. In the melted state, most perturbations of the configuration of the system are accepted by the algorithm. Analogously, a system that has a temperature of zero, which indicates that the algorithm cannot move to any higher-error state, is said to be \textit{frozen}. Note that a frozen system may still be perturbed into a lower-energy state.

When considering only the influence of classical thermal fluctuations in particle energy levels, the probability of a particle traversing a barrier of height \begin{math} \Delta V \end{math} at a temperature \begin{math} T \end{math} is on the order of: \begin{equation} 
\mathcal{P}_t = e^{-\frac{\Delta V}{T}} 
\end{equation}

% Be sure to include some discussion of thermal fluctuations and their importance. Find a nice source...
\subsection{Reheating}

\subsection{Application of Simulated Annealing to ANN Synaptic Weight Selection}

% Need sources for this... 

\section{Related Works in Quantum Mechanics}


Quantum mechanics is the branch of physics concerned with the physical laws of nature at very small scales. There are many aspects of physical reality than are observable only at these scales. Several techniques described in this document are either inspired by, or are simple models of quantum mechanical processes. These concepts are very briefly reviewed in this section. 

\subsection{Quantum Tunneling} 



One of the quantum phenomena for which there is no classical analog is potential barrier penetration, also known as quantum tunneling. This phenomenon arises from the probabilistic and wavelike behavior of particles in quantum physics. Tunneling plays a significant role in the behavior of bound and scattering quantum mechanical systems.

A particle with energy \begin{math} E \end{math} incident upon a potential energy barrier of height \begin{math} \Delta V > E  \end{math} has a non-zero probability of being found in, or past, the barrier. Classically, this behavior is forbidden. The probability of tunneling, \begin{math} \mathcal{P}_t \end{math}, through a step barrier of height \begin{math} \Delta V  \end{math} is described by: 
\begin{equation}
\mathcal{P}_t = e^{-\frac{w \sqrt{\Delta V}}{ \Gamma}} 
\end{equation} where \begin{math} \Gamma \end{math} is the tunneling field strength [Ref: Multivariable Opt: QAC - Mukherjee]. Figure [1] depicts a one-dimensional example of quantum tunneling.

[Figure 1] 



\subsection{Quantum Annealing}

Quantum annealing is the use of quantum, rather than thermal fluctuations to traverse the free energy landscape of a system. This is accomplished by introducing an additional Hamiltonian term that does not commute with the classical Hamiltonian. This non-commutation implies that [What does it mean?... The non-commutative causes the quantum effects...but how?].  The term is introduced to account for the presence of a tunneling field which controls the frequency with which quantum fluctuations occur in the system. In effect, this term controls the relative importance of quantum effects on the behavior of the modeled system. This term, much like thermal energy in simulated annealing, is gradually reduced over the course of the simulation. [Par Source: Quantum annealing in a kinetically constrained system] The time dependent Schrödinger equation \footnote{Note that the presence of the Schrödinger equation in section does not imply that quantum annealing requires the annealed system must be an approximation to a wavefunction. It merely serves as an exposition of the properties of physical system which is modeled.} for such a system has the form: \begin{equation}
[\lambda(t)H' + H_0]\psi = i\hbar \frac{\partial \psi}{\partial t}
\end{equation} [Eq from Mult Opt... Murherjee] where \begin{math} \lambda(t) \end{math} is the time-variance function of the tunneling field, \begin{math} H' \end{math} is the Hamiltonian term describing the tunneling field, and \begin{math} H_0 \end{math} is the classical Hamiltonian. 

The fluctuations induced by the tunneling field are tunneling events, which transition the system from one configuration to a different, lower-energy configuration directly, without assuming any of the higher energy configurations between the two. Said differently, the quantum tunneling field enables the penetration of energy barriers. The addition of these quantum fluctuations also ensures that each possible state of the system can be reached (KCS Paper). 

% Strictly speaking, this document does not claim to describe a quantum annealing process as it is presented in the referenced literature. 

% There is a great deal of academic writing describing QA in the language of physics, but very little writing describing the concept from an algorithmic perspective. For this reason, and because there are significant differences between the artificial simulation of quantum-inspired annealing and the physical phenomenon which is being approximated, a new terminology is proposed. In this document, a new more specific term is introduced in order to disambiguate.



% Simulated quantum annealing (SQA) is the quantum mechanical counterpart of simulated thermal annealing. Like simulated thermal annealing, simulated quantum annealing is a global search algorithm which seeks the global minimum of a cost function in a configuration space. 

% In continuously oscliating annealing schedules, consistently good states will tend to attract the search. Bad states will be randomly visited. Partition the space of states into for each synapse into discrete bins. Track the number of times a bin is hit by the search for each synapse. Given enough time and suffiently small bins, the modes of each histogram will come to represent the global minima.


\section{Notation and Terminology Conventions}

There is a great deal of academic writing describing quantum annealing in the language of physics, but very little writing describing the concept from an algorithmic perspective. For this reason a new, more specific term is introduced in this document. Simulated quantum annealing (SQA) is the quantum mechanical counterpart of simulated thermal annealing. 

% [Physics to Algorithmic translation table]
% Free Energy Surface - Error Manifold - Cost Function
% System Configuration -  






\chapter{Methodology}

\section{Simulated Quantum Annealing}

It is instructive to contrast equations 2.1 and 2.2. Both describe the same value, but the importance of the width and height of the traversed barrier in the two equations is considerably different. For systems in which quantum tunneling is possible, the probability of penetrating a barrier of height \begin{math} \Delta V \end{math} is increased by a factor of approximately \begin{math} e^{\Delta V} \end{math}, for large values of \begin{math} \Delta V \end{math}. This relationship is depicted graphically in Fig. 2 which shows the probability of barrier traversal for a system which allows quantum fluctuations, divided by the same probability for a system which only considers thermal fluctuations. Therefore, physical models which considers quantum effects are much more likely predict penetration of tall, thin energy barriers than those which only include classical thermal effects.


[Figure 2] 


\section{Traversing the Error Manifold}

\subsection{Unidimensional Weight Perturbation}

\subsection{Omni-dimensional Weight Perturbation}

\subsection{Constrained Step-Size Omni-dimensional Weight Perturbation}

\subsection{Quantum Weight Perturbation}


It is shown in Proof [n] that this algorithm is certain to eventually find the minimum possible

%formal proof of the eventual optimality of the simulated quantum annealing algorithm

\subsection{Stochastic-Anisotropic Quantum Weight Perturbation}



\subsection{Quantum Annealing}


[After discussing the way in which the algorithm is implemented] ...The net effect of this design is to allow the algorithm to move from a local minima configuration, to a different, lower-error configuration, without requiring the evaluation of intervening, higher-error configurations. This means that the probability of "tunneling" to a state




\chapter{Results}
\chapter{Conclusion}


\appendix		% Appendix begins here

\chapter{First appendix title}

\section{In an appendix} 

This is appendix section A.1.

Note: I highly recommend you create each chapter in a separate file
including the \verb|\chapter| command and \verb|\include| the file.
Then you can use \verb|\includeonly| to process selected chapters and
you avoid having to latex/preview/print your entire document every
time.

%\begin{Bibliography}	     % CAUTION: the first B is capital B.
%\bibitem[key] A listing ... % You can also use the `thebibliography'
%\bibitem[key2] A another    % environment described in LaTeX manual.
%\bibitem[key3]...	     % The usages of \bibitem and \cite{..} are
%\end{Bibliography}          % explained in Section 4.3 of the LaTeX manual.

% you can also use BibTeX instead of the above as I have done below. 
% see the LaTeX manual and the
% documentation available from /usr/TeX/doc.  There is an AFIT 
% bibliography style called thesnumb.  It has some special types 
% and fields.  See some sample entries and info in thesnumb.doc.  
% Note: thisthesis bibliography style only works with bibtex 
% version .99a or higher.

\bibliographystyle{thesnumb}
\bibliography{mybib1,mybib2,...}

\begin{vita}
	Insert your brief biographical sketch here. Your permanent
	address is generated automatically.
\end{vita}

%\begin{vita} %uncomment for twoauthor option
%	The second vita.
%	Insert the second authors brief biographical sketch here. 
%\end{vita}

\end{document}

% Please mail your suggestions and complaints to jdyoung@afit.af.mil.
